%-------------------------------------------------------------------------------
%	SECTION TITLE
%-------------------------------------------------------------------------------
\cvsection{Projects}


%-------------------------------------------------------------------------------
%	CONTENT
%-------------------------------------------------------------------------------
\begin{cventries}

%---------------------------------------------------------
	\cventry
		{Developer (Ruby on Rails, Bootstrap, jQuery)} % Role
		{MyToDo \textmd{\em\tiny(\url{https://github.com/indocomsoft/cvwo-assignment-2017})}} % Name of Project
		{} % Location
		{Dec 2017 - Jan 2017} % Date(s)
		{
			\begin{cvitems} % Description(s) of experience/contributions/knowledge
				\item {Developed a mobile-optimised To-Do List web application, complete with search function and categorisation.}
				\item {Set up automated CircleCI build, Coveralls code coverage, and CodeClimate maintanability assessment, and deployed on Heroku.}
				\item {Linted with rubocop, testing with rspec with 100\% code coverage and 9.73 hits/line, grade A for maintanability from CodeClimate.}
			\end{cvitems}
		}
		
	\cventry
	{Developer (Ruby on Rails, Bootstrap)} % Role
	{NUS Student Union Committee for IT Duty Website \textmd{\em\tiny(\url{http://github.com/commit-tech/takoyaki/})}} % Name of Project
	{} % Location
	{Jan 2017 - PRESENT} % Date(s)
	{
		\begin{cvitems} % Description(s) of experience/contributions/knowledge
			\item {Rewriting from PHP to Ruby on Rails the duty website used to track and manage computer lab supervisor duty.}
			\item {Redesigned and implemented the duty timetable interface with mobile-first principle to make it responsive and easier to read.}
			\item {Set up and wrote tests for models.}
		\end{cvitems}
	}
%---------------------------------------------------------
% \cventry
%    {Developer} % Affiliation/role
%    {NUSSU commIT Internal Website} % Organization/group
%    {Singapore} % Location
%    {Aug 2017 - PRESENT} % Date(s)
%    {
%      \begin{cvitems} % Description(s) of experience/contributions/knowledge
%      	\item {The website is used to manage supervising duty for members of the CCA and track tickets for issues with computers in the computer centres.}
%      	\item {Rewriting the CCA website in Ruby on Rails (done with pair programming). The legacy system utilised a buggy and difficult to maintain in-house framework.}
%      \end{cvitems}
%    }
    
%---------------------------------------------------------
\cventry
    {Developer (Elixir, Phoenix Framework, TypeScript, React, Redux)} % Role
    {Source Academy \textmd{\em\tiny(\url{https://github.com/source-academy/})}} % Name of Project
    {Singapore} % Location
    {Jan 2018 - PRESENT} % Date(s)
    {
      \begin{cvitems} % Description(s) of experience/contributions/knowledge
      	\item {Ironing out bugs in the back-end of the website and interpreter engine of The Source (a subset of Javascript focussing on its functional programming paradigm), the language used in CS1101S, a module based on MIT 8.001 course and the famous SICP book.}
      	\item {Developing new features for the website, including collaborative editing for its IDE and debugger for its interpreter engine.}
      \end{cvitems}
    }
%---------------------------------------------------------

\cventry
    {Builder and Developer (OpenWRT, Shell Script)} % Role
    {Wi-Fi Extender and Load-balancer \textmd{\em\tiny (\url{https://github.com/indocomsoft/Maker-Portfolio})}} % Name of Project
    {Singapore} % Location
    {Jan 2016 - Oct 2016} % Date(s)
    {
      \begin{cvitems} % Description(s) of experience/contributions/knowledge
      	\item {Built a system using Wi-Fi routers connected via powerline adaptors to extend coverage, but one Wi-Fi router is used as a load-balancer, utilising several other Wi-Fi routers to obtain higher speed on a Wi-Fi network with per-device speed limit.}
      	\item {Wrote a shell script for monitoring and generating traffic so as not to be logged out from the captive portal. Also wrote a script to parallelise rsync and downloads. More details available on the link above.}
      \end{cvitems}
    }
%---------------------------------------------------------
\end{cventries}
