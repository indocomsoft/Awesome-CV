%-------------------------------------------------------------------------------
%	SECTION TITLE
%-------------------------------------------------------------------------------
\cvsection{Projects}


%-------------------------------------------------------------------------------
%	CONTENT
%-------------------------------------------------------------------------------
\begin{cventries}

%---------------------------------------------------------
	\cventry
		{Developer (Ruby on Rails, Bootstrap, jQuery)} % Role
		{MyToDo (\url{https://github.com/indocomsoft/cvwo-assignment-2017})} % Name of Project
		{} % Location
		{Dec 2017 - PRESENT} % Date(s)
		{
			\begin{cvitems} % Description(s) of experience/contributions/knowledge
				\item {A mobile-optimised To-Do List web application, complete with categorisation.}
				\item {Set up automated CircleCI build, Coveralls code coverage, and CodeClimate maintanability assessment, and deployed on Heroku.}
			\end{cvitems}
		}
		
%---------------------------------------------------------
% \cventry
%    {Developer} % Affiliation/role
%    {NUSSU commIT Internal Website} % Organization/group
%    {Singapore} % Location
%    {Aug 2017 - PRESENT} % Date(s)
%    {
%      \begin{cvitems} % Description(s) of experience/contributions/knowledge
%      	\item {The website is used to manage supervising duty for members of the CCA and track tickets for issues with computers in the computer centres.}
%      	\item {Rewriting the CCA website in Ruby on Rails (done with pair programming). The legacy system utilised a buggy and difficult to maintain in-house framework.}
%      \end{cvitems}
%    }
    
%---------------------------------------------------------
\cventry
    {Developer (Elixir, Phoenix Framework, TypeScript, React, Redux)} % Role
    {Source Academy} % Name of Project
    {Singapore} % Location
    {Jan 2018 - PRESENT} % Date(s)
    {
      \begin{cvitems} % Description(s) of experience/contributions/knowledge
      	\item {The website for CS1101S, a module based on the famed SICP book, taught in a subset of JavaScript called The Source.}
      	\item {It is used as an IDE for the language used in the module, as well as a platform to disseminate problem sets, lecture notes, etc. and to collect and grade assignments.}
      \end{cvitems}
    }
%---------------------------------------------------------

\cventry
    {Builder and Developer (OpenWRT, Shell Script)} % Role
    {Wi-Fi Extender and Load-balancer (\url{https://github.com/indocomsoft/Maker-Portfolio})} % Name of Project
    {Singapore} % Location
    {Jan 2016 - Oct 2016} % Date(s)
    {
      \begin{cvitems} % Description(s) of experience/contributions/knowledge
      	\item {Using Wi-Fi routers connected via powerline adaptors to extend coverage from one point to the other, but one Wi-Fi router is used as a load-balancer, utilising several other Wi-Fi routers in order to obtain higher speed on a Wi-Fi network with per-device speed limit.}
      	\item {Wrote a shell script for monitoring and to generate traffic so as not to be logged out from the captive portal. Also wrote a script to parallelise rsync and downloads. More details available on the link above.}
      \end{cvitems}
    }
%---------------------------------------------------------
\end{cventries}
